\documentclass[a4paper, 10pt]{article}

\usepackage[top=2.54cm, bottom=2.86cm, left=2cm, right=2cm]{geometry}
\usepackage{multicol}

\usepackage{booktabs}
\usepackage{multirow}

\usepackage{graphicx}

\usepackage[sortcites=false,natbib=true,            style=ieee,backend=biber,maxbibnames=3,giveninits=true,uniquename=false,uniquelist=true,parentracker=true,url=false,doi=false,isbn=true,eprint=false,backref=true]{biblatex}
\addbibresource{references.bib}

\title{PA2552 Software Metrics - Assignment 1}
\author{
    Vassilis Asteriou -- \textit{vaat18@student.bth.se} -- \textit{981023T093}
    \\
    Claudio Bertozzi -- \textit{clbt18@student.bth.se} -- \textit{930206T578}
    \\
    Pranavi Bitra -- \textit{prbi18@student.bth.se} -- \textit{9710027443}
    \\
    Jules Lecuir -- \textit{julc18@student.bth.se} -- \textit{960509T330}
}

\date{\today}

\begin{document}

    \maketitle
    \begin{multicols}{2}
    
        \begin{abstract}
            In this document, we extract data from research papers and report different metrics we come across in the papers, providing a description and categorization of them. We also review two software metrics extraction tools and give our feedback about how we used them. 
        \end{abstract}
        
        \section{Introduction}
        
            Metrics are a crucial tool for the improvement of software development processes. In this assignment, our goal is to get familiar with different object oriented (OO) metrics and tools and discuss them. 
            
            First, we will extract information from papers about software metrics and analyze information from ten of them, that focus on measuring coupling, cohesion, maintainability and understandability. We summarize them in section \ref{article_summaries}. Then, we will cover the metrics, systems under study, metrics extraction tools and visualization tools as well as statistical methods we came across in this literature review in sections \ref{categorization_of_metrics}, \ref{categorization_of_SUS}, \ref{metrics_extraction_tools} and \ref{visualizzation_statistical_tools}. Secondly, we will evaluate two different software metrics tools by using them on an open source project in section \ref{review_tools_software_metrics}. There, we will discuss their ease of installation and use, the metrics that they compute, and their visualization and data exporting features.
            
        \section{Article summaries}
\label{article_summaries}
\subsection{Summaries of articles about cohesion metrics}
    \subsubsection{\citetitle{s118_cohesion} \cite{s118_cohesion}}
    
        The authors investigate four different metric types to find a valid metric for cohesion at design level. In their understanding, the similarity of the parameter types of the different methods in a class indicates that they process closely related information. In this case the calculated metric value is height. I the class has a low cohesion the value is close to 0.

        The used metric types are CAMC, NHD, SNHD and NHDM. All those metrics are based on mathematical calculations from the parameter-occurrence matrix which represents the different methods with their parameter types. 
        
        For the reason that all metric types have different side effects regarding anomalies or dependencies on the size of the class, the results for the cohesion value variate strongly. An interesting fact they could find out is that the NHDM always shows the lowest cohesion value and eliminates most of the side effects. They substantiate this to the fact that a class is not cohesive in general.


    \subsubsection{\citetitle{s12_cohesion} \cite{s12_cohesion}}
    \label{article_summary_s12}
    
        The article introduces a cohesion metric called Low-level Similarity-based Class Cohesion metric (LSCC).
        
        Until now, existing cohesion metrics like LCOM1, LCOM2, LCOM3, LCOM4, TCC, LCC, DC$_D$, DC$_I$ have several flaws, LSCC aims to solve most of them. LSCC is the only one that has a strong mathematical basis and makes it a more coherent and well-defined metric. It considers the number of shared attributes in Method-Method-Interactions (MMIs), and supports class inheritance and transitive interactions between methods. It captures a new dimension of cohesion on its own, as shown by the statistical studies. LSCC has been empirically validated regarding its relationship with other external metrics, such as fault occurrences.
        
        Thus, LSCC seems to be an outstanding metric for cohesion. Moreover, a mathematically compliant framework for automatic refactoring advice is provided with the metric. Thanks to a generic formula the need to execute \textit{Move Method} or \textit{Extract Class} activities can be evaluated automatically.
        
        However, LSCC is not semantically\,---\,i.e. only structurally\,----\,based and cannot capture this aspect of cohesion. The framework can help to refactor but still cannot differ if an attribute should better be deleted rather than moved\,----\. Moreover, the calculus for refactoring takes computation time and has to be executed for every couple of methods, and cannot be applied in some huge classes with many methods.
        
    \subsubsection{\citetitle{s29_cohesion} \cite{s29_cohesion}}
    
        The authors develop a method for applying the \textit{Extract Class} refactoring to classes with low cohesion, based on structural and semantic cohesion metrics. They construct a class-method graph, weighted with a composite cohesion measure, and then, use the \textit{MaxFlow-MinCut} algorithm to find the two result classes.
        
        The authors investigate the effects that the operation parameters have on the resulting metrics and whether the combination of structural and semantic measures of cohesion results in improved outcome relative to using only one type. Their case study involves merging class pairs into blobs, performing the refactoring and evaluating the similarity to the original class pair using the F-measure. They repeat that for three software projects and various values of the parameters to spot the optimal configuration and using statistical tests, they answer positively regarding the relevance of both types of measures.

        Finally, the authors test their method against real blobs analyzed in a previous publication. They note that the refactoring operation yields lower cohesion and, based on postgraduate student opinions, sometimes reasonable results, but it fails in cases where the operation is inappropriate, which is reflected in coupling metrics.
    
\subsection{Summaries of articles about maintainability metrics}

    \subsubsection{\citetitle{s116_maintainability} \cite{s116_maintainability}}
    
        Maintainability in software systems is defined as probability of performing essential changes to increase the quality and performance of the product. The purpose of this research is to form a software maintainability prediction model using data mining on existing maintainability metrics, and datasets of open-source softwares and data mining classifiers. The authors want to give an accurate maintainability prediction model which calculates the probability of change in different modules of the code, and thus helps lessens maintenance efforts.
        
        The classifiers are Naïve Bayes, Bayes Network, Logistic Regression, Multilayer perceptron and Random Forest. It is found that Random forest models are accurate software maintainability prediction model which are determined by data mining of metrics and have high performance measures regarding recall, precision, and ROC area. The new software metrics like Halstead bugs (B), CLOC, Command, Inner, Dcy are useful for prediction of software maintainability. The limitation is that the process of selection of predictor variables which have an impact on maintainability and is specific to particular application domains. Thus, a generic selection of variables cannot be created, because it depends on the context of application.
    
    \subsubsection{\citetitle{s219_maintainability} \cite{s219_maintainability}}
    \label{article_summary_s219}
    
        This article focuses on the link between occurrences of code smells in the code and their effects on the maintainability of a project. The purpose is to know if maintenance problems can be predicted by the automatic detection of code smells. The authors conducted a case study on four real softwares, designed in the context of the scientific experiment, for a \textit{unique purpose}, and which are thus \textit{functionally equivalent} although they differ in size and internal architecture. After detecting code smells, the authors conducted a scientific study on the application of the same maintenance task on these four softwares and report maintenance problems encountered during the experiment.    
        
        The authors acknowledge the limits of this study: first, the definition of a \textit{maintenance problem} can be subject to interpretation, even if a clear definition is given in this particular article. The analysis were mainly qualitative and manually executed, leading to subjective interpretation, even if solid triangulation approaches have been used to reduce biases.
        
        The authors' observation is that code smells relate for (only) 30\% of maintenance problems. Most of the code smells are subject to interaction effects with other code smells, or other parts of the code. Most of them could already be spotted through other indicators such as coupling, size and complexity. Thus, code smells detection alone cannot furnish a global vision on future maintenance problems and has to be combined with other metrics.
    
    \subsubsection{\citetitle{s13_maintainability} \cite{s13_maintainability}}
    
        In this paper, the authors use a Mamdani fuzzy logic engine with the objective of predicting the number of changes that happen in each step of the project maintenance phase based on several known maintainability metrics. While discussing related work, they make the point that fuzzy methods can be an approach well suited for measuring software, precisely because it is so multi-faceted. As their working data, they use Li and Henry's datasets of metrics from the UMIS and QUES software projects. They make use of Pearson’s Correlation Coefficient to identify which of the variables present in the data bear a relation to the changes to the software. Based on the result, they construct two different fuzzy logic predictors that they train using the datasets. After that, they evaluate the model by comparing the actual and predicted changes and by computing relative error statistics, noting the good performance of the prediction model.

\subsection{Summaries of articles about coupling metrics}

    \subsubsection{\citetitle{s88_coupling} \cite{s88_coupling}}
    
        The authors investigated into static and dynamic coupling metrics regarding the correlation in between. Therefore, they selected one static object-oriented metric type called coupling between objects (CBO) and its equivalent for dynamic measurements called Dynamic CBO (DCBO). CBO is counting the coupled classes for a specific class. DCBO does nothing else than to measure the same but during runtime. For this reason, the code must be manipulated so that the dynamic coupling data can be gathered. This mechanism is called dynamic profiling.
        
        For the investigation they used different Java open-source applications which they first selected by specific criteria to reduce biases. After applying the metric, they compared the results from CBO to DCBO. Most of the applied metrics on the test applications share a weak to moderate correlation. That leads to the conclusion, that static and dynamic coupling metrics do not capture exactly the same aspects of coupling.

    \subsubsection{\citetitle{s23_coupling} \cite{s23_coupling}}
    
        The context of this paper is to select a case study of three types of software, and consider coupling measures between components to find design patterns. The objective is to choose the most significant coupling measure that should be used for a project and which shows the good interactions between the components in the object-oriented software environment. 
        
        Main findings are results obtained from the experiment of the object class feature vectors considering two principal components having range of total variance representing the entire dataset and it considers the most significant component of software which has no uniform coupling measure. The types of software utilize metrics like TNUCC, Class coupling, RNUCC, RNUCD, AHF. Limitations are relating the coupling measures to the software applications tasks and determine which can be used for it. The analysis of the object-oriented system has inconsistencies regarding existing measures which cannot be represented in a unified framework.
    
    \subsubsection{\citetitle{s89_coupling} \cite{s89_coupling}}
    
        The authors define and evaluate empirically a coupling metric using the Relational Topic Model, a variant of the Latent Dirichlet Allocation generative model. The outputs of this procedure are the link indicator probabilities, which measure the coupling between all pairs of classes, and which can be aggregated to quantify the coupling of a single class to the other classes of the software. 
        
        To evaluate the metric, the authors perform a case study on various C++/Java open-source software projects of various sizes. They conduct a Principal Component Analysis, suggesting that, compared to previously known coupling metrics, RTC covers a new principal component accounting for a different $\approx$ 8\% of coupling from either structural or conceptual coupling metrics, i.e. which measures a different aspect of coupling than other metrics. Furthermore, in the context of impact analyses, they argue that RTC outperforms structural coupling in both precision and recall statistics, and that the same statistics are improved when RTC is used in combination with conceptual metrics.
    
\subsection{Summaries of articles about understandability metrics: \citetitle{s68_understandability} \cite{s68_understandability}}

        Understandability is essential for projects to explore the relationships between packages that group classes and evaluate software quality of object-oriented designs. The authors conduct a case study on two open-source software systems with a suite of five package level metrics and perform correlation, collinearity and multivariate regression analyses. The objective is to measure the properties of packages like size, coupling, and stability by testing null hypotheses using metrics like NC, Ca, Ce, I, D to evaluate the prediction models that determine the effort to understand a package.
        
        The results indicate statistically significant positive correlation among all metrics and understandability of package (except a negative correlation with Ce). The results of the collinearity analysis show that, by neglecting I metric in prediction models, the other metrics are not affected. THe limitations lie in the multivariate regression analysis, which considers only package size for prediction models. The data to measure package understandability was not a reliable study as the it contains dataset having 18 packages.

        
        \section{Object-oriented metrics used in the papers}
\label{categorization_of_metrics}

    The different types of metrics we found in the papers are summarized in tables \ref{table:cohesion_metrics} for cohesion metrics, \ref{table:coupling_metrics} for coupling metrics, \ref{table:maintainability_metrics} for maintainability metrics, and \ref{table:understandability_metrics} for understandability metrics.
    
    The coupling metrics we encountered are either structural, i.e. they are based on the structure of the program, or conceptual/semantic, i.e. they are based on natural language used in comments and identifiers. We decided to classify cohesion metrics based their ability to consider the degree of interaction between methods, i.e. not only classifying MMIs in a binary (and thus limited) way, but rather in a more fine-grained way. Maintainability metrics were split into two groups, as they are either process- or code-related. Understandability metrics are categorized as package level based metrics.
    
    \begin{table*}[t]
    \centering
    
    \begin{tabular}{p{2cm} p{4cm} p{10cm}}
    
        \toprule
        
            \textbf{Subtype of metric} & \textbf{Name} & \textbf{Description} \\
            
        \midrule
        
            \multirow[c]{3}{\linewidth}{\tiny\\\normalsize Considering the degree of interaction between methods} & Low-level Similarity-based Class Cohesion (LSCC) & LSCC is the average cohesion of all pairs of methods. Calculated through the MAR (Method-Attribute Reference) matrix that accounts for transitive interactions caused by method invocations, it considers the ratio of similarity between methods in a mathematically compliant and fine-grained way \cite{s12_cohesion}.\\
                \cmidrule(rl){2-3}& Class Cohesion (CC) & Ratio of the summation of the similarities between all pairs of methods to the total number of pairs of methods \cite{s12_cohesion, s29_cohesion}.\\
                \cmidrule(rl){2-3}& Class Cohesion Metric (SCOM) & Same as CC, but with a different equation to measure the similarity \cite{s12_cohesion}.\\
                \cmidrule(rl){2-3}& Call-based Dependence between Methods (CDM) & Method-method metric equal to the proportion of incoming calls to a method that originate from a specific method. Being an asymmetric measure, the maximum value is used as a commutative measure \cite{s29_cohesion}. \\
                
        \midrule
        
            \multirow[c]{8}{\linewidth}{Not considering the degree of interaction between methods} & Structural Similarity between Methods (SSM) & Given two methods, the SSM is the proportion of instance variables that both methods use out of all used instance variables in those methods \cite{s29_cohesion}.\\
                \cmidrule(rl){2-3}& Lack of Cohesion of Methods (LCOM or LCOM1) & Defined as the number of pairs of methods that do not share attributes \cite{s12_cohesion, s13_maintainability, s29_cohesion, s116_maintainability}. \\
                \cmidrule(rl){2-3}& LCOM2 & Defined as the substraction of the number of pairs of methods that do not share attributes by the number of pairs of methods that share attributes. Equals zero if the result is negative \cite{s12_cohesion, s29_cohesion}.\\
                \cmidrule(rl){2-3}& LCOM3 & In a graph where each node is a method of the class and each edge is one or more attributes shared between two methods, LCOM3 is equal to the number of edges \cite{s12_cohesion}.\\
                \cmidrule(rl){2-3}& LCOM4 & In the LCOM3 graph, we add additional edges for method invocations, and count the edges again \cite{s12_cohesion}.\\
                \cmidrule(rl){2-3}& Tight Class Cohesion (TCC) & Relative number of directly connected pairs of methods, where two methods are directly connected if they are directly connected to an attribute  \cite{s12_cohesion}.\\
                \cmidrule(rl){2-3} & Loose Class Cohesion (LCC) & Relative number of indirectly connected pairs of methods, where two methods are indirectly connected if they are directly or indirectly connected to an attribute  \cite{s12_cohesion}.\\
                \cmidrule(rl){2-3}& Degree of Cohesion - Direct (DC$_D$) & Relative number of directly connected pairs of methods, where two methods are directly connected if they satisfy TCC or if they directly or transitively invoke a same method  \cite{s12_cohesion}. \\
                \cmidrule(rl){2-3}& Degree of Cohesion - Indirect (DC$_I$) & Relative number of directly or transitively connected pairs of methods, where two methods are transitively connected if they satisfy LCC or if they directly or transitively invoke a same method \cite{s12_cohesion}.\\
                \cmidrule(rl){2-3}& Cohesion among Methods in a Class (CAMC) & Measures the extent of intersection of individual method parameter \cite{s118_cohesion}. \\
                \cmidrule(rl){2-3}& Normalized Hamming Distance (NHD) & Calculate the cohesion based on the definition of hamming distance regarding the individual method parameter \cite{s118_cohesion}. \\
                \cmidrule(rl){2-3}& Scaled NHD (SNHD) & Interprets the metrics from NHD in a more varied range \cite{s118_cohesion}. \\
                \cmidrule(rl){2-3}& NHD modified (NHDM) & This metric focuses on removing the anomalies affecting the metrics NHD and SNHD \cite{s118_cohesion}. \\
                
        \bottomrule
        
    \end{tabular}
    
    \caption{Description of the cohesion metrics encountered in the studied papers.}
    \label{table:cohesion_metrics}
    
\end{table*}
    \begin{table*}[t]
    \centering
    \begin{tabular}{p{2cm} p{4cm} p{10cm}}
        \toprule
            \textbf{Subtype of metric} & \textbf{Name} & \textbf{Description} \\
        \midrule
            \multirow{6}{*}{Structural} 
             & Coupling Between Object classes (CBO)	& Sometimes defined as the number of classes referenced in a class, sometimes as the number of classes either referenced or referencing a class \cite{s89_coupling, s29_cohesion, s116_maintainability}.\\
            \cmidrule(rl){2-3} & Dynamic CBO (DCBO) & DCBO measures similar aspects of coupling like CBO. For the Dynamic CBO, the metric data will be gathered during run-time and in a second step, the metric numbers are calculated \cite{s88_coupling}. \\
            \cmidrule(rl){2-3} & Response For Class (RFC) & Number of methods either belonging to the class or called in one of the class methods \cite{s89_coupling, s13_maintainability, s116_maintainability}.  \\
            \cmidrule(rl){2-3} & Message Passing Coupling (MPC) & Number of method calls from one class to another \cite{s89_coupling, s13_maintainability, s29_cohesion, s116_maintainability}. \\
            \cmidrule(rl){2-3} & Data Abstraction Coupling (DAC) & Number of instance variables having an abstract data type \cite{s89_coupling, s13_maintainability}.\\
            \cmidrule(rl){2-3} & Briand et al. measure suite (including ACAIC, OCAIC, ACMIC, OCMIC) & General number of interactions of other classes elements with the class measured. In the mentioned cases, counts of the references to a class' attributes (CA) and methods (CM) differentiated between references from ancestors (A) and other (O) classes in the hierarchy \cite{s89_coupling}. \\
        \midrule
            \multirow[p]{2}{*}{Conceptual} 
            & Conceptual Similarity between Methods (CSBM, CSBM\textsubscript{m}) and Conceptual Coupling of Classes (CoCC, CoCC\textsubscript{m})  & Latent  Semantic  Indexing based coupling metrics. After parsing, indexing and representation of methods as vectors in the semantic space, their angle cosine is calculated. These metrics are aggregated (with average or maximum) over method pairs to obtain metrics on a class pair basis (CSBM and CSBM\textsubscript{m} resp.) and then over classes to obtain a class metric (CoCC and CoCC\textsubscript{m} resp.) \cite{s89_coupling}.\\
            \cmidrule(rl){2-3} & Relational Topic - based Coupling (RTC\textsubscript{S}, RTC\textsubscript{C}) & Link probabilities between methods in a class, based on the Relational Topic Model. Per class metric can be calculated by averaging over the class' methods \cite{s89_coupling}. \\
          
         \midrule
            \multirow[p]{2}{*}{Without Ratio} & Total number of evidences for “User classes through dependency relation” (TNUCC) & Metric that measures the count of total number of usage evidences of a class by the other classes \cite{s23_coupling}. \\
             \cmidrule(rl){2-3} & Class Coupling (CLC) & Metric that measures how one class is connected or dependent with another class. The sum estimation of Client and Server Coupling of class considers out and in degree of a node in Class Class Interaction Graph (CCIG)  \cite{s23_coupling}.\\
            
        \midrule
            \multirow[p]{4}{*}{With Ratio} & Ratio of NUCC to TNUCC (RNUCC) & Metric that measures the ratio between TNUCC and NUCC only when count of NUCC is greater than 1  \cite{s23_coupling}. \\
             \cmidrule(rl){2-3} & Attribute Hiding Factor (AHF) & Metric that defined as ratio of sum of invisibilities of all attributes in classes to total no. of attributes in the system  \cite{s23_coupling}.\\
        
        \bottomrule
    \end{tabular}
    \caption{Description of the coupling metrics encountered in the studied papers.}
    \label{table:coupling_metrics}
\end{table*}    \begin{table*}[t]
    \centering
    \begin{tabular}{p{2cm} p{4cm} p{10cm}}
        \toprule
            \textbf{Subtype of metric} & \textbf{Name} & \textbf{Description} \\
            
            \midrule
            \multirow{2}{\linewidth}{\tiny\\\normalsize Process based} & Halstead Bugs (B) & Metric that finds no. of bugs in a class in the implementation and correlated with complexity of software \cite{s116_maintainability}. \\
               \cmidrule(rl){2-3} & Weighted Methods per Class (WMC) and Number of Methods (NoM) & The WMC of a class is the sum of the complexities of its methods, usually in terms of McCabe's cyclomatic complexity. When all methods are considered to have a complexity of 1, then the WMC is the number of methods \cite{s13_maintainability, s116_maintainability}. \\ 
            
            \midrule
            
            \multirow{6}{\linewidth}{Code based} & Depth in the Inheritance Tree (DIT)	& Distance of a class from the root of the inheritance tree \cite{s13_maintainability}. \\
            
                \cmidrule(rl){2-3} & Number of Children (NoC) & Number of direct subclasses of a  class.\cite{s13_maintainability} \\
            
                \cmidrule(rl){2-3} & SIZE1 and SIZE2 & SIZE1 is the number of semicolons in a class (e.g. Java). SIZE2 is the number of attributes and local methods of a class \cite{s13_maintainability}. \\
            
                \cmidrule(rl){2-3} & Comments Lines of Code (CLOC) 	& Metric that measures number of lines of code that has comments in each class \cite{s116_maintainability}.  \\
            
                \cmidrule(rl){2-3} & Number of transitive dependencies (Dcy) & Metric that measures no. of classes or interfaces that are directly or indirectly dependent on each class \cite{s116_maintainability}. \\
    
                 \cmidrule(rl){2-3} & Number of Line of code changed (CHANGE) & Metric that counts the change in source code as addition of one line as one, deletion as one, modification as two for determining the total change that occurred \cite{s13_maintainability, s116_maintainability}. \\ 
        \bottomrule
        
    \end{tabular}
    
    \caption{Description of the maintainability metrics encountered in the studied papers.}
    \label{table:maintainability_metrics}
    
\end{table*}
    \begin{table*}[t]
    \centering
    \begin{tabular}{p{4cm} p{12cm}}
        \toprule
            \textbf{Name} & \textbf{Description} \\
            
            \midrule
            Number of classes (NC) 	& Metric that measures package size which determine no. of concrete and abstract classes \cite{s68_understandability}. \\
            \midrule Afferent Couplings(Ca) & Metric that measures incoming dependencies for package which determine no. of other packages that depend on classes within package \cite{s68_understandability}. \\
            \midrule Efferent Coupling (Ce)	& Metric that measures outgoing dependencies for package which determine no. of other packages that classes in package depend upon \cite{s68_understandability, s116_maintainability}. \\
            \midrule
            Instability (I) 	& It is ratio of efferent coupling to total coupling. Total coupling is Afferent and efferent coupling \cite{s68_understandability}. \\
            \midrule
            Distance (D)	& It measures perpendicular distance of package from idealized line which indicates package balance between abstractness and stability \cite{s68_understandability}. \\
            
        \bottomrule
    \end{tabular}
    \caption{Package-level metrics used for quantifying understandability.}
    \label{table:understandability_metrics}
\end{table*}
    
\section{Systems studied in the papers}
\label{categorization_of_SUS}
    
    \begin{description}
        \item[Chatting softwares:] Turtle Chat (chatting software using Internet Protocol with server and client Part) \cite{s23_coupling}; Com Chat (chatting software using Java Communication API) \cite{s23_coupling}.
        
        \item[Text and visual editors:] Ekit (scenario based editor) \cite{s88_coupling}; JHotDraw (two-dimensional graphics framework) \cite{s89_coupling, s116_maintainability}; JEdit (Java-based text editor) \cite{s116_maintainability};  VoodooUML (Tcl/Tk-based UML editor) \cite{s89_coupling}; Umbrello (KDE-based UML editor) \cite{s89_coupling}; AgroUML (Java-based UML editor) \cite{s29_cohesion}; GanttProject (project management application) \cite{s12_cohesion}. 
        
        \item[Database management software:] Lucene (Java-based indexing and information retrieval software library) \cite{s116_maintainability}; Admission Test Management System (database software to store university students) \cite{s23_coupling}; Library Management System (scenario based database application) \cite{s88_coupling}; Charting library (Java open-source software for visualizing data) \cite{s118_cohesion}; JabRef (graphical application for bibliographical databases) \cite{s12_cohesion}.
        
        \item[3D modelling software:] Art Of Illusion (3D modeling and animation application) \cite{s12_cohesion}; Rhino (3D modelling, analysis, documentation, rendering and animation of 3D objects) \cite{s89_coupling}.
        
        \item[Text generation software:] XGen Source Code Generator (creates text output from structured text input) \cite{s68_understandability}; Jakarta Element Construction Set (Java API generating elements for various markup languages like HTML and XML) \cite{s68_understandability}.
        
    \end{description}
    
    We found other softwares in the papers that we didn't find pertinent to categorize, either because we lack information about them, such as Quality Evaluation Software (QUES) and User Interface Management System (UIMS) in \cite{s13_maintainability}, or because some are alone in their own categories, such as Openbravo (point-of-sale application designed for touch screens) \cite{s12_cohesion}, SPECjvm2008 (Java Virtual Machine Benchmark) \cite{s88_coupling}. In \cite{s219_maintainability}, the systems under study are web applications that were created all in the context of the scientific study, but no name is provided for them.
    
\section{Metrics extraction tools used in the papers}
\label{metrics_extraction_tools}

        \begin{itemize}
            
            \item CohMetric is a computational tool that produces indices for written and spoken texts. It measures cohesion and coherence metrics that allow readers to instantly gauge the difficulty of written text for the target audience \cite{s118_cohesion}.
            
            \item In \cite{s12_cohesion}, the metrics extraction tool is a home-made Java program that computes eleven cohesion metrics and automates the statistical comparison between them. 
            \item In \cite{s29_cohesion}, the automatic refactoring of classes is also performed by a home-made tool.
            
            \item Chidamber and Kemerer Java Metric (CKJM) is the program that calculates the object-oriented metrics by processing the bytecode of compiled Java files. IntelliJ IDEA is an IDE meant for Java development \cite{s116_maintainability}.
           
            \item The tools  Borland Together and InCode were used to extract automatically code smells from source code, along with manual analysis of maintenance issues, such as daily interviews, screen recordings, code review, etc \cite{s219_maintainability}.
            
            \item Jerry Mendel's fuzzy logic tool is used to codify the fuzzy logic engine \cite{s13_maintainability}.
             
            \item The static metric evaluator parses the bytecode of application and reports the static metrics values using javassit library. The dymanic coupling data is collected by dynamic profiling \cite{s88_coupling}.
            
            \item Design Analyzer is developed in Java to analyze the source code of object-oriented system to retrieve the design patterns. It shows the interactions, relationships, roles, collaboration in graphical format \cite{s23_coupling}.
            
            \item To measure coupling, Columbus, IRC\textsuperscript{2}M and R/lda are used. Columbus is a reverse-engineering-based tool for C++ programs. IRC\textsuperscript{2}M is a tool for measuring semantic coupling metrics. The lda package in the R language is used for Latent Dirichlet Allocation \cite{s89_coupling}.
            
            \item In \cite{s68_understandability}, no tool was mentioned by the authors.
            
        \end{itemize}
        
        
\section{Visualization methods and statistical measures used in the papers}
\label{visualizzation_statistical_tools}
    
    \subsection{Methods to visualize extracted metrics}
    
        Various data and concept visualization methods can be found throughout the literature. Tables are commonly used for a variety of data and result visualization purposes, including displaying the rotating components of Principal Component Analysis \cite{s23_coupling, s89_coupling}, evaluating prediction models \cite{s68_understandability, s116_maintainability} and summarizing details and statistics \cite{s219_maintainability, s12_cohesion, s118_cohesion, s88_coupling, s13_maintainability, s116_maintainability}. 
        
        Other visualization methods include tree maps and pie charts to display distributions \cite{s219_maintainability, s12_cohesion}, box plots and bar charts to visualize comparisons and results \cite{s88_coupling, s13_maintainability, s29_cohesion}, diagrams to plot dependent against independent variables \cite{s29_cohesion}, a machine learning training history diagram \cite{s13_maintainability}, a Venn diagram describing a hierarchy \cite{s219_maintainability} and graphs describing methodologies or algorithms employed \cite{s12_cohesion, s88_coupling, s29_cohesion}.
        
    \subsection{Statistical measures and methods}
        
        There are many statistical approaches in the literature to verify results and evaluate proposed methodologies. Many papers include tables of descriptive statistics \cite{s68_understandability, s12_cohesion, s118_cohesion, s88_coupling, s13_maintainability}, usually including the minimum, maximum, mean and standard deviation, for a preliminary outlook over a dataset. 
        A frequently seen method is the Principal Component Analysis, which helps explain the variance of a data set in terms of orthogonal components \cite{s23_coupling, s12_cohesion, s89_coupling}.
        Correlation and tests are also frequently employed. Spearman's correlation coefficient, which identifies relations of monotony, is used in correlation and collinearity analyses \cite{s68_understandability, s12_cohesion}, whereas Pearson's correlation coefficient, which identifies linearity, is also used to analyze correlation \cite{s88_coupling, s13_maintainability}.
        
        To determine whether samples come from identical distributions the non-parametric Wilcoxon tests are applied -  specifically, the Wilcoxon Rank Sum test \cite{s29_cohesion}, which examines independent samples, and the Wilcoxon Signed Rank test \cite{s89_coupling} which examines dependent samples.
        Regression Analysis, which models and analyzes the effect of one or more independent variables on a dependent variable, is used in different forms, including Multivariate Regression Analysis \cite{s68_understandability, s12_cohesion} and Univariate Logistic Regression \cite{s12_cohesion}.
        In evaluation of the developed metrics and methods the Precision and Recall statistics, their composition into the F-measure and the Receiver Operation Characteristic are used, which quantify the performance of binary classification techniques \cite{s116_maintainability, s12_cohesion, s89_coupling}.
        Other measures used include the Mahalanobis distance, which measures the distance of a point from a distribution \cite{s12_cohesion}, and various error metrics including the Root Mean Square Error and its normalized variant (RMS, NRMS) and the Mean Magnitude of Relative Error (MMRE) \cite{s68_understandability, s13_maintainability}.
        
        % \begin{itemize}
        %     \item The cohesion metric averages and standard deviations in descriptive statistics while conducting empirical analyses \cite{s118_cohesion}.
        %     \item The LSCC metric is a ratio, so any statistical measure can be performed on it, such as the mean, median, standard deviation, confidence intervals, and more The other metrics are not mathematically compliant, as explained in the summary of the article (see \ref{article_summary_s12}), so no proper statistical measure can be made on them. However, a correlation and loading matrix as well as a logistic regression were conducted to compare LSCC to the other metrics \cite{s12_cohesion}.

        %     \item The Pearson Correlation Coefficient is a statistical measure of the strength of correration of two continuous variables, i.e. how dependent they are on one another \cite{s116_maintainability, s13_maintainability}.
            
        %     \item As stated in the summary in section \ref{article_summary_s219}, the study focused mainly on qualitative aspects of the link between code smells and maintenance problems. No statistical study has been deeply conducted during the study, only counting of defects and sometimes ratios \cite{s219_maintainability}.

        %     \item Principal Component Analysis is a technique for identifying uncorrelated variables from large data set, based on the eigen decomposition of the covariance matrix. It is linear transformation technique that can help emphasize variation and show strong patterns in a dataset \cite{s23_coupling, s89_coupling} .

        %     \item For independent and dependent variables utilize descriptive statistics to summarize dataset by min, max, average, standard deviation.
        %     For correlation analysis it considers Spearman’s rank correlation is a non parametric version that measures the strength and direction of association between two ranked variables.
        %     For collinearity it considers Variance Inflation Factor (VIF) that quantifies the correlation between independent variables and estimates variance of a regression coefficient is inflated due to collinearity in the model.
        %     For multivariate Regression analysis it considers Mean Magnitude of relative error (MMRE) and Prediction level at 0.25. \cite{s68_understandability}. 
        % \end{itemize}
       

    
        \section{A review of two selected software metrics tools}
\label{review_tools_software_metrics}

To get used to working with software metrics tools, we chose two different tools that we used to apply different metric types on a Java open-source project.

\subsection{iText}

Our group chose the project iText. We used the current version of iText which is iText 7. This is a Java framework for manipulating PDF files. The project which can be found on GitHub includes 12 different projects which are built into different \verb|.jar| files.

\subsection{Our first tool: Metrics 1.3.8}

The first tool we chose is an Eclipse plugin called Metrics.
    
    \subsubsection{Ease of installation and use}
    
        At first we tried to install the Metrics 1.3.6 plugin on Eclipse but that resulted in class loader exceptions, so we tried to install the version 1.3.8 instead. The installation process takes place entirely inside the IDE and it is straightforward and well documented on the plugin's web page. 
        
        Once the tool is installed, one can select a package in the package explorer, enable Metrics in the Properties dialog and see the metrics in the Metrics View. The metrics are recomputed with every build. However, the metrics can just be applied for a single project selected in the package explorer.
        
        The configuration is as expected in the program preferences under a proper node. The settings are easy to find and what they adjust is clear. However, there are only a  few settings, which leads to the conclusion that a lot of metrics can not be parametrized.

    \subsubsection{OO metrics coverage}
    
        Several mainstream metrics are computed by this tool, including McCabe's Cyclomatic Complexity and Weighted Methods per Class, Depth in the Inheritance Tree, Number of Classes/Interfaces/Packages/Methods/Attributes, Number of Children, LOC (non-comment, non-blank lines), MLOC (LOC in each method) and LCOM. Furthermore, Afferent and Efferent Coupling (Ca, Ce) along with Instability (I) are reported.
        
        All the metrics are calculated at a package level as well as at a class level. For some metrics it even provides the metrics at method level.
    
    \subsubsection{Reporting possibilities}
    
        The tool reports the results in a tree-view element, where the user can inspect the values of the metrics along with basic statistics (total, maximum, average, standard deviation) on the package, class and method level. Within one level of abstraction, the values of the metrics are presented in the order of worse to better, and values over a specified limit are highlighted to indicate a problematic piece of code to the user. The user can also obtain XML reports of the results. Finally, there is also a dependency graph visualization option but it suffers from critical usability issues.
    
\subsection{Our second tool: Understand}

    \subsubsection{Ease of installation and use}
    
        The tool is easy to install, \textit{via} an executable on Windows. It offers a complete user interface with different tools and visualization tabs.
        
        Creating a new metrics project is simple and once we select the main folder of our software project, all the metrics are calculated at every levels: method, class, package, and system level. Overall, the tool has good understandability and learnability properties : by wandering around around in the software, we discovered easily the different export options for reporting and visualization. The online documentation is also exhaustive.
        
        The backside in this software is that we don't know exactly where to search when we want to control the parameters of the metrics and understand them. We would have liked more parametrization possibilities in the metrics computation.
        
        It is also worth noting that we tried the trial version of the software. Understand is a commercial software with different pricings.
    
    \subsubsection{OO metrics coverage}
    
        The software covers the main metrics such as many common metrics derived from the counting of lines (LOC, CLOC, NCLOC, SLOC), complexity metrics (average and maximum Cyclomatic Complexity, maximum nesting). It also offers derived metrics such as the Comment/Code ratio. It also offers class OO metrics like LCOM, DIT, IFANIN, CBO, NOC, RFC, NIM, NIV, WMC. The tool gives numeric data for each class in the project regarding the metrics.
    
    \subsubsection{Reporting possibilities}
    
        Understand offers different ways to export metrics and create reports. 
        
        \begin{itemize}
            \item We can first ``export the metrics'' about the project in a single CSV file. Options are granted to select or unselect all the metrics we want to export. We have the choice between a CSV export and an HTML export. The CSV file gives the metrics (when available) for each entity that the software can manage: method, class, package,system. The file is huge and need to be refined in a spreadsheet editor if we want to extract something consistent from it. The HTML option allows to visualize data in a tree view and access metrics at any level of the system, even for different programming languages, but removes the possibility to make further use of the data (which is possible with the CSV file).
            
            \item We can also generate another, even more ``visually friendly'' report. It offers pie and bar charts, as well as tables referencing all the metrics available for any entity of the system. Although the user interface and the reports are more attractive, the charts lacks an exhaustive legend. In the dashboard view for example, the units are missing on every pie chart and bar chart. It made our understanding ambiguous and confused us a bit.
        
        \end{itemize}
        
        Overall, the software has great capabilities for data reporting. It also has some integrated tools to visualize class dependencies, and can create tree maps with personalized metrics, which can be useful too.
    
        
        \section{Conclusion}
            
            In this assignment, we have deepened our knowledge on some of the most used metrics in the scientifical sphere. We familiarized ourselves with software metrics tools and could discuss their possibilities and limitations.
            
    \end{multicols}
    
    \clearpage
    \printbibliography

\end{document}